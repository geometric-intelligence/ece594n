\documentclass{beamer}

\usetheme{default} 
\usepackage{graphicx}
\usepackage{amsmath}
\usepackage{amssymb}
\usepackage{mathtools}
\usepackage{hyperref}
\hypersetup{
    colorlinks=true,
    linkcolor=blue,
    filecolor=magenta,      
    urlcolor=blue
}

\newcommand{\R}{\mathbb{R}}
\newcommand{\Z}{\mathbb{Z}}
\newcommand{\norm}[1]{\left\lVert#1\right\rVert}


\title{Group Equivariant Convolutional Networks}
\author{Daniel Ralston}
\date{1/22/2024}

\begin{document}

% Title Slide
\begin{frame}
  \titlepage
\end{frame}

% Table of Contents
\begin{frame}{Agenda}
    \begin{itemize}
        \item Introduction and Motivation
        \item Related Works
        \item Background: Groups, Group Actions, and Equivariance
        \item Methods
        \item Results
        \item Demonstration of the Code
        \item Conclusion
    \end{itemize}
\end{frame}

% Introduction
\section{Introduction}
\begin{frame}{Introduction}
    \begin{itemize}
        \item Purpose of the research/study
        \item Importance and implications
        \item Brief overview of the approach
    \end{itemize}
\end{frame}

% Related Works
\section{Related Works}
\begin{frame}{Related Works}
    \begin{itemize}
        \item Overview of existing research in the area
        \item Comparative analysis with previous studies
    \end{itemize}
\end{frame}

% Background
\section{Background}
\begin{frame}{Groups}
    \begin{itemize}
        \item A \emph{group} is a set $G$ with a binary operation $\cdot$ such that:
        \begin{enumerate}
            \item $G$ is closed under $\cdot$
            \item $\cdot$ is associative
            \item There exists an identity element $e \in G$ such that $e \cdot g = g \cdot e = g$ for all $g \in G$
            \item For each $g \in G$, there exists an inverse $g^{-1} \in G$ such that $g \cdot g^{-1} = g^{-1} \cdot g = e$
        \end{enumerate}
        \item In this paper, the authors focus on groups of rigid transformations of the plane (e.g. subgroups of $SE(2)$, the 2-dimensional special Euclidean group)
    \end{itemize}
\end{frame}

\begin{frame}{$p4$ and $p4m$}
    \begin{itemize}
        \item $p4$ -- all 2-dimensional integer translations and rotations by multiples of $\frac{\pi}{2}$
        \begin{itemize}
            \item The underlying set can be described as a set of matrices where $r \in \{0, 1, 2, 3\}$ and $u, v \in \mathbb{Z}$
            $$g(r, u, v) = \begin{bmatrix}
                \cos(\frac{\pi}{2}r) & -\sin(\frac{\pi}{2}r) & u \\
                \sin(\frac{\pi}{2}r) & \cos(\frac{\pi}{2}r) & v \\
                0 & 0 & 1
            \end{bmatrix}$$
        \end{itemize}
        \item $p4m$ -- all 2-dimensional integer translations, rotations by multiples of $\frac{\pi}{2}$, and mirror reflections
        \begin{itemize}
            \item For $r \in \{0, 1, 2, 3\}$, $u, v \in \mathbb{Z}$, and $m \in \{0, 1\}$
            $$g(m, r, u, v) = \begin{bmatrix}
                (-1)^m\cos(\frac{\pi}{2}r) & (-1)^{m+1}\sin(\frac{\pi}{2}r) & u \\
                \sin(\frac{\pi}{2}r) & \cos(\frac{\pi}{2}r) & v \\
                0 & 0 & 1
            \end{bmatrix}$$
        \end{itemize}
        \item In both cases, the binary operation is matrix multiplication
    \end{itemize}
\end{frame}

\begin{frame}{Group Actions}
    %mention how a group (R, +) acts on a vector space by scalar multiplication
    \begin{itemize}
        \item Critical to this paper, the authors use the fact that these groups act on the set of images.
        \item A group $G$ is said to act on a set $X$ if there exists a function $\gamma: G\times X \to X$ such that 
        $$\gamma(e, x) = x \text{ ($e$ is the identity element of $G$)}$$
        $$\gamma(g_1, \gamma(g_2, x)) = \gamma(g_1g_2, x)$$
        \item $p4$ and $p4m$ act on $\Z^2$ (specifically $\Z^2 \times \{1\} \subset \R^3$) by matrix-vector multiplication:
        \begin{itemize}
            \item Ex: For $A\in p4$ and $[u, v, 1]^T \in \Z^2 \times \{1\}$, $$A[u, v, 1]^T = [u', v', 1]^T \in \Z^2 \times \{1\}$$
        \end{itemize} 
    \end{itemize}
\end{frame}

\begin{frame}{Acting on the set of images}
    \begin{itemize}
        \item The authors describe the set of images as the collection of functions $f: \Z^2 \to \R^K$ (where $f$ has compact (rectangular) support)
        \item $p4$ and $p4m$ act on the the set of images $\{f\}$ with the function $L$:
        $$L_g(f)(x) = f(g^{-1}x)$$
        where $g^{-1}x$ denotes matrix-vector multiplication
    \end{itemize}
\end{frame}

\begin{frame}{Equivariance}
    \begin{itemize}
        \item A function $f: X \to Y$ is said to be equivariant with respect to the action of $G$ on $X$ and $Y$ if
        $$f(\gamma(g, x)) = \gamma(g, f(x))$$
        \item In this paper, the authors are interested in functions $f: \{f\} \to \{f\}$ that are equivariant with respect to the action of $p4$ and $p4m$ on $\{f\}$
        \item The authors define a convolutional layer to be equivariant if the function $f$ is equivariant with respect to the action of $p4$ and $p4m$ on $\{f\}$
    \end{itemize}
\end{frame}
% Methods
\section{Methods}
\begin{frame}{Methods}
    \begin{itemize}
        \item Detailed description of the methodology used
        \item Explain the algorithms, models, and tools
    \end{itemize}
\end{frame}

% Results
\section{Results}
\begin{frame}{Results}
    \begin{itemize}
        \item Presentation of the key findings and outcomes
        \item Statistical analysis and data interpretation
        \item Discuss the figures and tables from the paper that showcase the results
    \end{itemize}
\end{frame}

% Demonstration of the code
\section{Demonstration of the Code}
\begin{frame}{Demonstration of the Code}
    \begin{itemize}
        \item Walkthrough of the main segments of the code
        \item Execution of the code and presentation of results in real-time (if possible)
    \end{itemize}
\end{frame}

% Conclusion
\section{Conclusion}
\begin{frame}{Conclusion}
    \begin{itemize}
        \item Summarize the main points from the presentation
        \item Implications of the results 
        \item Possible directions for future work
    \end{itemize}
\end{frame}

% References
\begin{frame}{References}

\end{frame}


\end{document}